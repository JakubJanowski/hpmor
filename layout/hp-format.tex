% This file includes all the generic formatting for HPatMoR. This mostly entails configuring
% the memoir package, though “configuring” on occasion means “completely messing it up”.

\RequirePackage{fmtcount}

% Fonts used generally (specific fonts used only once or twice are not here).
\defaultfontfeatures{Ligatures={TeX}}
\setmainfont[
  Extension=.otf
, UprightFont=*-Regular
, ItalicFont=*-Italic
, BoldFont=*-Bold
, BoldItalicFont=*-BoldItalic
, SmallCapsFont=AlegreyaSC-Regular
]{Alegreya}

\newfontface\hpfont[ExternalLocation]{Lumos}
\newfontface\ptsansi[ExternalLocation]{AlegreyaSans-Italic}

% Drop-caps at start of chapters
\renewcommand{\LettrineFontHook}{\hpfont}
\renewcommand{\LettrineTextFont}{}
\renewcommand{\DefaultLoversize}{.2}
\renewcommand{\DefaultLraise}{0.1}

% Extend lettrine cutout when more than one paragraph goes alongside the drop-cap
% Copied, with different macro names, from
% https://tex.stackexchange.com/questions/369868/using-lettrine-with-short-paragraphs
\newcount\hplettrinecount
\makeatletter
\def\hplettrineextrapara{%
\ifnum\prevgraf<\c@L@lines
\hplettrinecount\z@
\loop
\ifnum\hplettrinecount<\prevgraf
\advance\hplettrinecount\@ne
\afterassignment\hplettrinedima\count@\L@parshape\relax
\repeat
\parshape\L@parshape
\fi}
\def\hplettrinedima{\afterassignment\hplettrinedimb\dimen@}
\def\hplettrinedimb{\afterassignment\hplettrinedef\dimen@}
\def\hplettrinedef#1\relax{\edef\L@parshape{\the\numexpr\count@-1\relax\space #1}}
\makeatother
\newcommand{\lettrinepara}[3][]{\lettrine[nindent=0pt,#1]{#2}{#3}}

% Allow linebreaks after em-dash and hyphens, except when they’re followed by punctuation
\newXeTeXintercharclass \punctuationClass

\XeTeXcharclass `\’ \punctuationClass
\XeTeXcharclass `\‘ \punctuationClass
\XeTeXcharclass `\“ \punctuationClass
\XeTeXcharclass `\” \punctuationClass
\XeTeXcharclass `\. \punctuationClass
\XeTeXcharclass `\, \punctuationClass
\XeTeXcharclass `\: \punctuationClass
\XeTeXcharclass `\? \punctuationClass
\XeTeXcharclass `\! \punctuationClass
\XeTeXcharclass `\: \punctuationClass

\newXeTeXintercharclass \digitClass
\XeTeXcharclass `\0 \digitClass
\XeTeXcharclass `\1 \digitClass
\XeTeXcharclass `\2 \digitClass
\XeTeXcharclass `\3 \digitClass
\XeTeXcharclass `\4 \digitClass
\XeTeXcharclass `\5 \digitClass
\XeTeXcharclass `\6 \digitClass
\XeTeXcharclass `\7 \digitClass
\XeTeXcharclass `\8 \digitClass
\XeTeXcharclass `\9 \digitClass

\newXeTeXintercharclass \dashClass
\XeTeXcharclass `\— \dashClass % em
\XeTeXcharclass `\– \dashClass % en
\XeTeXcharclass `\- \dashClass % hyphen
\XeTeXcharclass `\… \dashClass

\XeTeXinterchartokenstate = 1

\def\morhyphenpenalty{75}
\exhyphenpenalty=10000

\XeTeXinterchartoks \dashClass 0 = {\hskip 0pt\penalty \morhyphenpenalty}

% Adjust space around lists
\setlength{\topsep}{.5\baselineskip plus 1\baselineskip minus .5\baselineskip}
\setlength{\partopsep}{.5\baselineskip plus 1\baselineskip minus .5\baselineskip}

% Miscellaneous global typesetting parameters
\frenchspacing % Make spacing after all punctuation the same as the spacing between words
\raggedbottom
\linespread{1.125}

% Prefer underfull to overfull lines
\tolerance=260 % Allow more horizontal spacing between words, default = 200
\hbadness=5000 % When compiling, only reports underfull lines with badness > 4000
\hfuzz=0.35pt
\setlength{\emergencystretch}{2.125em} % Maximum allowed space width on an already overfull line
%\overfullrule=10mm % Mark overfull lines in generated PDF with 10mm black rectangles (for debugging only)

% Try to avoid excessive hyphens
\doublehyphendemerits=30000 % Avoids hyphenation on consecutive lines
\finalhyphendemerits=30000
\adjdemerits=10000
\brokenpenalty=10000 % for a page break, where the last line of the previous page contains a hyphenation.

% Avoid widows and orphans
\clubpenalty=10000 % Avoids first line of paragraph on previous page
\widowpenalty=10000 % Avoids last line of paragraph on next page

% Various packages used
\usepackage[normalem]{ulem}
\usepackage{censor}
\usepackage[useregional]{datetime2}
\usepackage{gitinfo2}
\usepackage[shortcuts]{extdash}
\usepackage{comment}
